% SYLLABUS ----
% This is just here so I know exactly what I'm looking at in Rstudio when messing with stuff.
\documentclass[11pt,]{article}
\usepackage[margin=1in]{geometry}
\newcommand*{\authorfont}{\fontfamily{phv}\selectfont}
\usepackage[]{mathpazo}
\usepackage{abstract}
\renewcommand{\abstractname}{}    % clear the title
\renewcommand{\absnamepos}{empty} % originally center
\newcommand{\blankline}{\quad\pagebreak[2]}

\providecommand{\tightlist}{%
  \setlength{\itemsep}{0pt}\setlength{\parskip}{0pt}}
\usepackage{longtable,booktabs}

\usepackage{parskip}
\usepackage{titlesec}
\titlespacing\section{0pt}{12pt plus 4pt minus 2pt}{6pt plus 2pt minus 2pt}
\titlespacing\subsection{0pt}{12pt plus 4pt minus 2pt}{6pt plus 2pt minus 2pt}

\titleformat*{\subsubsection}{\normalsize\itshape}

\usepackage{titling}
\setlength{\droptitle}{-.25cm}

%\setlength{\parindent}{0pt}
%\setlength{\parskip}{6pt plus 2pt minus 1pt}
%\setlength{\emergencystretch}{3em}  % prevent overfull lines

\usepackage[T1]{fontenc}
\usepackage[utf8]{inputenc}

\usepackage{fancyhdr}
\pagestyle{fancy}
\usepackage{lastpage}
\renewcommand{\headrulewidth}{0.3pt}
\renewcommand{\footrulewidth}{0.0pt}

\lhead{}
\chead{}
\rhead{\footnotesize Research Automation -- Verano 2021}
\lfoot{}
\cfoot{\small \thepage/\pageref*{LastPage}}
\rfoot{}

\fancypagestyle{firststyle}
{
\renewcommand{\headrulewidth}{0pt}%
   \fancyhf{}
   \fancyfoot[C]{\small \thepage/\pageref*{LastPage}}
}

%\def\labelitemi{--}
%\usepackage{enumitem}
%\setitemize[0]{leftmargin=25pt}
%\setenumerate[0]{leftmargin=25pt}




\makeatletter
\@ifpackageloaded{hyperref}{}{%
\ifxetex
  \usepackage[setpagesize=false, % page size defined by xetex
              unicode=false, % unicode breaks when used with xetex
              xetex]{hyperref}
\else
  \usepackage[unicode=true]{hyperref}
\fi
}
\@ifpackageloaded{color}{
    \PassOptionsToPackage{usenames,dvipsnames}{color}
}{%
    \usepackage[usenames,dvipsnames]{color}
}
\makeatother
\hypersetup{breaklinks=true,
            bookmarks=true,
            pdfauthor={ ()},
             pdfkeywords = {},
            pdftitle={Research Automation},
            colorlinks=true,
            citecolor=blue,
            urlcolor=blue,
            linkcolor=magenta,
            pdfborder={0 0 0}}
\urlstyle{same}  % don't use monospace font for urls


\setcounter{secnumdepth}{0}





\usepackage{setspace}

\title{Research Automation}
\author{Marco Antonio Ramos}
\date{Verano 2021}

\usepackage{tikz}

\newcommand{\shrug}[1][]{%
\begin{tikzpicture}[baseline,x=0.8\ht\strutbox,y=0.8\ht\strutbox,line width=0.125ex,#1]
\def\arm{(-2.5,0.95) to (-2,0.95) (-1.9,1) to (-1.5,0) (-1.35,0) to (-0.8,0)};
\draw \arm;
\draw[xscale=-1] \arm;
\def\headpart{(0.6,0) arc[start angle=-40, end angle=40,x radius=0.6,y radius=0.8]};
\draw \headpart;
\draw[xscale=-1] \headpart;
\def\eye{(-0.075,0.15) .. controls (0.02,0) .. (0.075,-0.15)};
\draw[shift={(-0.3,0.8)}] \eye;
\draw[shift={(0,0.85)}] \eye;
% draw mouth
\draw (-0.1,0.2) to [out=15,in=-100] (0.4,0.95);
\end{tikzpicture}}

\linespread{1.05}

\begin{document}

		\maketitle
	

		\thispagestyle{firststyle}

%	\thispagestyle{empty}


	\noindent \begin{tabular*}{\textwidth}{ @{\extracolsep{\fill}} lr @{\extracolsep{\fill}}}


E-mail: \texttt{\href{mailto:marcoyel21@gmail.com}{\nolinkurl{marcoyel21@gmail.com}}} & Web: \href{http://github.com/marcoyel21}{\tt github.com/marcoyel21}\\
Office Hours: Por definir  &  Class Hours: Por definir\\
Office: San Pedro 189, Col: Del Cármen  & Class Room: \emph{online}\\
	&  \\
	\hline
	\end{tabular*}

\vspace{2mm}



\hypertarget{descripciuxf3n-del-curso}{%
\section{Descripción del curso}\label{descripciuxf3n-del-curso}}

En este curso se aprenderan técnicas, buenas prácticas y \emph{know how}
sobre automatización, replicabilidad y colaboración en equipo en la
investigación. En general, son practicas que emanan de las ciencias
computacionales y que se han comenzado a utilizar en la investigación
económica debido a los beneficios que conllevan.

\begin{itemize}
\item
  La replicabilidad de las investigaciones es una buena práctica que
  cada vez tiene mayor importancia. La transparencia en las limpiezas de
  datos, estimación de modelos y obtención de resultados le brinda mayor
  credibilidad a la investigación además de que diciplina al
  investigador a ser ordenado, a documentar cada paso y a minimizar las
  deciciones arbitrarias durante su proceso de investigación.
\item
  La automatización de los pasos en el proceso de investigación permiten
  al investigador ahorrarse el tiempo y minimizar la cantidad de errores
  que se cometen al darle formato al documento, a las tablas, a las
  ecuaciones, al citar en determinado formato, al insertar figuras,
  notas al pie, indices y demás utilidades estéticas. Asimismo, en
  conjunto con la replicabilidad, la automatización implementada de
  manera exitosa garantiza que se obtendran los mismos resultados de
  manera automática si se corre el \emph{script} desde cualquier
  computadora.
\item
  Finalmente, la colaboración en equipo de manera remota es un área con
  muchisimo potencial de utilidad en el contexto de la pandemia, la
  globalización y de la evolución de la manera de trabajar y colaborar
  en la investigación tanto para la academía como para las empresas.
  Permite organizar de manera rigurosa el proceso de investigación,
  repartir las distintas tareas y crear un ambiente de colaboración
  virtual en el que todos los integrantes pueden trabajar sobre el mismo
  proyecto sin riesgo de desincronización de los avances y duplicación
  de tareas.
\end{itemize}

El curso esta diseñado para las personas que hacen investigación, ya
sean alumnos de posgrado o investigadores de alguna institución.
Asimismo, el curso es de particular interés para todas aquellas personas
que quieren integrar su limpieza, modelado y redacción en un solo
documento; que quieren mejorar su dominió de las herramientas para la
investigación econométrica; que han batallado mezclando avances en WORD
o PPTX con los de STATA, R o Python; que han perdido horas valiosas en
las citas y referencias; para aquellos que siguen usando recortes para
mostrar tablas y ecuaciones; para aquellos que han tenido problemas de
gestión colaborando en equipo de manera remota; que han tenido problemas
en cuanto a la estética de su investigación o presentación; para
aquellos que gestionan grandes y pequeños equipos de investigación; y en
general, para todos aquellos que quieren ahorrarse tiempo y esfuerzo en
el proceso de investigación.

\hypertarget{objetivos-del-curso}{%
\section{Objetivos del curso}\label{objetivos-del-curso}}

\begin{enumerate}
\def\labelenumi{\arabic{enumi}.}
\item
  Que ustedes puedan automatizar su ciclo de trabajo personal o en
  equipo con el fin de que se dejen de preocupar en los formatos, citas,
  matemáticas, referencias, tablas, etc y se concentren al 100\% en sus
  investigaciones
\item
  Que dominen una variedad de herramientas para que realmente plasmen el
  potencial de la investigación que pensaron inicialmente
\item
  Que logren crear una investigación replicable (o bien replicar un
  \emph{pape}) en equipo con su respectivo documento, limpieza,
  presentación, bibliografía, etc.
\end{enumerate}

\hypertarget{requisitos}{%
\section{Requisitos}\label{requisitos}}

\begin{itemize}
\tightlist
\item
  Conocimiento básico del lenguaje de programación R
\end{itemize}

\hypertarget{froma-de-trabajo}{%
\section{Froma de trabajo}\label{froma-de-trabajo}}

\begin{itemize}
\item
  El curso consiste en 3 horas de contenido diarios más dos talleres el
  fin de semana: a) para crear un proyecto de investigación
  (colaborativo, automático y replicable) y; b) para crear un CV o
  Syllabus (automático, estético y replicable).
\item
  El curso comprendera 1 o 2 semanas (por definir)
\item
  Modalidad a distancia
\end{itemize}

\hypertarget{contenido}{%
\section{Contenido}\label{contenido}}

\hypertarget{latex-y-markdown}{%
\subsection{Latex y Markdown}\label{latex-y-markdown}}

\begin{itemize}
\item
  \emph{Operaciones matemáticas}
\item
  \emph{Tablas}
\item
  \emph{Jerarquía de textos}
\item
  \emph{Figuras avanzadas}
\end{itemize}

\hypertarget{r-markdown}{%
\subsection{R Markdown}\label{r-markdown}}

\begin{itemize}
\item
  \emph{Code chunks}
\item
  \emph{Funcionalidades esteticas (indice, encabezados, pies de pagina,
  numeración y referencias)}
\item
  \emph{Utilidades en el YAML (automatización de procesos dentro de
  RMD)}
\item
  \emph{Llamar variables locales}
\item
  \emph{Replicabilidad y buenas prácticas}
\end{itemize}

\hypertarget{formatos-de-rmd}{%
\subsection{Formatos de RMD}\label{formatos-de-rmd}}

\begin{itemize}
\item
  \emph{HTML}
\item
  \emph{PDF}
\item
  \emph{PPTX}
\end{itemize}

\hypertarget{mendeley-automatizaciuxf3n-de-citas-y-referencias}{%
\subsection{Mendeley: automatización de citas y
referencias}\label{mendeley-automatizaciuxf3n-de-citas-y-referencias}}

\begin{itemize}
\item
  \emph{El formato BIB}
\item
  \emph{Administrador de citas}
\item
  \emph{El formato CSL}
\item
  \emph{Integración a RMD}
\end{itemize}

\hypertarget{github}{%
\subsection{Github}\label{github}}

\begin{itemize}
\item
  \emph{Comandos básicos en la terminal}
\item
  \emph{Configuración inicial}
\item
  \emph{Administra tus repositorios de manera local}
\item
  \emph{Colaboración en equipo a través de un repositorio}
\item
  \emph{Administración de proyectos con github (Issues, Milestones y
  Proyectos)}
\item
  \emph{Ramas, Merges, Pull requests y conflictos}
\end{itemize}

\hypertarget{ambientes-virtuales}{%
\subsection{Ambientes virtuales}\label{ambientes-virtuales}}

\begin{itemize}
\item
  \emph{Replicabilidad}
\item
  \emph{Conda}
\item
  \emph{Docker}
\end{itemize}

\hypertarget{costos}{%
\section{Costos}\label{costos}}

El costo total del curso es de 2,000 MXN y comprenderá una o dos semanas
dependiendo del \emph{quorum}, el interés y la intensidad.

\hypertarget{referencias}{%
\section{Referencias}\label{referencias}}

\href{https://bookdown.org/yihui/rmarkdown/}{R Markdown}

\href{https://r4ds.had.co.nz/index.html}{R for Data Science}

\href{https://github.com/miollek/Free-Git-Books/blob/master/book/Git\%20Pocket\%20Guide\%20-\%20A\%20Working\%20Introduction.pdf}{Git
Pocket Guide}

\newpage




\end{document}

\makeatletter
\def\@maketitle{%
  \newpage
%  \null
%  \vskip 2em%
%  \begin{center}%
  \let \footnote \thanks
    {\fontsize{18}{20}\selectfont\raggedright  \setlength{\parindent}{0pt} \@title \par}%
}
%\fi
\makeatother
